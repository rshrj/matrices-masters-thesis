\section{D0-brane Matrix Model}

Many great reviews exist on the subject, e.g., \cite{Taylor:2001vb,Bilal:1997fy,Ydri:2017ncg}. The M-theory matrix model or the BFSS model can be obtained through three different methods (that are, in fact, equivalent)
\begin{enumerate}
    \item[a.] Worldline theory of a large number ($N$) of D0 branes in type IIA string theory. \cite{Itzhaki:1998dd}
    \item[b.] Dimensional reduction of $9+1$ dimensional Super Yang-Mills (SYM) theory to $0+1$ dimensions \cite{Claudson:1984th}.
    \item[c.] Matrix regularization of the light-front quantized supermembrane theory \cite{deWit:1988wri}.
\end{enumerate}

The quantum mechanical model one obtains through this method has dynamical degrees of freedom that are $N \times N$ Hermitian matrices, and the structure of the theory is fixed entirely by three symmetries, $\mathrm{SU}(N)$, $\mathrm{SO}(9)$ and supersymmetry. In particular, one has $9$ bosonic matrices (as in, adjoint of $\mathfrak{su}(N)$) $X^i \ i=1,\ldots,9$ that transform as an $\mathrm{SO}(9)$ vector and $16$ fermionic matrices collectively called $\theta$ that transform as a $\mathrm{Spin}(9)$ Majorana spinor (each spinor component is an $N \times N$ matrix with grassmann odd elements). 

The matrices $X^i$ have the interpretation of the $N$ D0-brane positions when they are simultaneously diagonal. The off-diagonal entries of $X^i$ represent stringy degrees of freedom stretching from one D0-brane to another \cite{Polchinski:1996na}. The complete action one obtains through any of the above procedures (upto rescalings) is as follows
\begin{align}
    \begin{split}
    S[X, \theta, A] = \frac{1}{g_\mathrm{YM}^2} \int \text{d}t & \Tr \left( \frac{1}{2} D_{t}X^{i} D_{t}X^{i} \right. \\ 
    & \left. + \frac{1}{4} \comm{X^i}{X^j}\comm{X^i}{X^j} + \frac{1}{2} \theta^T (i D_t{\theta} - \gamma_i \comm{X^i}{\theta}) \right)
    \end{split} 
\end{align}
where the gauge covariant derivative acts as
\begin{equation}
    D_t X^i = \dot{X}^i - i \comm{A_t}{X^i}
\end{equation}

The parameter $g_\mathrm{YM}^2$ is the Yang-Mills coupling and has dimensions of $(\mathrm{energy})^3$. The relative coefficient between the bosonic and fermion parts of the action is fixed by supersymmetry. One can eliminate the $U(N)$ gauge field $A$ by fixing the gauge $A_t = 0$. Further, we restrict our attention to states in the model with no fermionic modes $\theta = 0$. Thus, the reduced theory is a matrix model of 9 Hermitian matrices $X^i$ with $i = 1, \ldots, 0$ with the action
\begin{equation}\label{eqn:reduced-bfss-action}
    S[X] = \frac{1}{2 g_\mathrm{YM}^2} \int \text{d}t \Tr \left( \dot{X}^i \dot{X}^i + \frac{1}{2} \comm{X^i}{X^j}\comm{X^i}{X^j} \right)
\end{equation}

% As it stands, the action in \cref{eqn:reduced-bfss-action} is invariant under the following scale transformation parameterized by $\lambda > 0$
% \begin{alignat}{2}
%     t &\to \tilde{t} && = \ \lambda t \\
%     X^i(t) &\to \tilde{X}^i(\tilde{t}) && = \ \lambda X^i(t) 
% \end{alignat}we

The equations of motion are given by
\begin{equation}\label{eqn:eom}
    \ddot{X}^i = -\comm{\comm{X^i}{X^j}}{X^j}
\end{equation}

The remnant of gauge fixing is the gauge constraint which follows from the $A_t$ equation of motion before gauge fixing.

\begin{equation}
    \comm{\dot{X}^i}{X^i} = 0
\end{equation}

A particularly nice way to visualize matrix configurations is to map them to embeddings of a sphere (or, more generally, of Riemann surfaces) in $\mathbb{R}^9$. This can be done for the sphere using matrix spherical harmonics (see \cref{app:matrix-harmonics}). We have chosen to exclude any results or further discussion of matrix harmonics in this regard since this would require truncating the matrix model to two or three matrices and would distract us from the primary purpose of this thesis. 

A curious consequence of the equations of motion \cref{eqn:eom} is that configuration of matrices that are block diagonal, i.e.
\begin{equation}
    X^i = \begin{pmatrix}
        A^i & \\
        & B^i
    \end{pmatrix}
\end{equation}
where $A^i$ and $B^i$ are submatrices, and the velocities $\dot{X}^i$ are also diagonal, then the equations of motion ensure that the block diagonal pieces evolve completely independently with no interactions. The interpretation is that separate clusters of D0-branes see no interactions in the classical matrix model. Indeed, a surprising and wonderful result of Matrix theory is that if one includes quantum corrections to the effective potential between such clusters of D0-branes to one loop, we recover the classical supergravity interaction \cite{Douglas:1996yp,Paban:1998ea,Kabat:1997sa}. Indeed, this was pointed out by BFSS as a concrete test of their conjecture. In particular, in the simplest situation where two D0-brane particles are separated by a distance $r$ and with a relative velocity $v$ between them, the quantum effective potential around this background turns out to be
\begin{equation}
    V = -\frac{15}{16}\frac{v^4}{r^7} + \mathcal{O}\left( \frac{v^6}{r^{11}} \right)
\end{equation}
to leading order in $v/r^2$. This is precisely the Newton-like classical potential that 11-dimensional supergravity between probe test particles reduces to in the quasi-static approximation (see \cite{Taylor:2001vb} for an excellent review of matrix model interactions).


\section{Scaling similarity}
The gauge fixed BFSS action (including fermions) enjoys a scaling similarity under which
\begin{align}\label{eqn:scaling-similarity}
    t \to \lambda^{-1} t \qquad X^i \to \lambda X^i \qquad \theta \to \lambda^{3/2} \theta
\end{align}
We call this transformation a similarity (and not a symmetry) because it does not keep the action invariant, but it changes by a factor
\begin{equation}
    S[X, \theta] \to \lambda^3 S[X, \theta]
\end{equation}
Hence, it is a symmetry of the equations of motion but not a symmetry of the full quantum theory. Since we will be doing classical simulations, we will exploit this similarity to normalize the energy surfaces we study to any positive energy, say, $E = 1$. We can always do this because the Hamiltonian
\begin{equation}
    H = \frac{1}{2 g_\mathrm{YM}^2} \Tr \left( \dot{X}^i \dot{X}^i - \frac{1}{2} \comm{X^i}{X^j}\comm{X^i}{X^j} + \theta^{T}\gamma_i \comm{X^i}{\theta} \right)
\end{equation}
under \cref{eqn:scaling-similarity} scales like
\begin{equation}
    H \to \lambda^4 H
\end{equation}

Similarly, observables homogenous in the $X^i$ such as $\Tr X^i X^i$ scale in a specific way under \cref{eqn:scaling-similarity}. e.g.
\begin{equation}
    \Tr X^i X^i \to \lambda^2 \Tr X^i X^i
\end{equation}
Thus, we restrict attention to observables of this type. Further, we rescale all our variables so that $g_\mathrm{YM} = 1$.

\section{Recent results}
Many efforts have been made to understand non-perturbative aspects of string/M-theory via dual Matrix models using numerical methods. These primarily focus on Monte Carlo-type simulations in Euclidean spacetime (see \cite{Hanada:2012eg,Anagnostopoulos:2007fw} for a review). What distinguishes our work from the papers just cited is that we are working in a regime of Matrix theory where there are enormous stringy corrections in the dual quantum gravity. Thus, no simple description or expectations exist on that side. Further, we do real-time classical simulations instead of periodic Euclidean time path integrals. Despite this, we find behavior reminiscent of ordinary black holes, such as random matrices and the typicality of microstates.

The Monte Carlo methods used in \cite{Hanada:2012eg} have been quite successful at doing precision tests of the holographic correspondence. For instance, the thermodynamic relations for the dual black geometry (see \cite{Maldacena:2018vsr}) have been derived numerically for large matrices where the result holds, including higher derivative corrections \cite{Hanada:2008ez}. Further, an interesting observable in these gauge theories that we will not have much to say about (since we are not working in Euclidean time) is the Polyakov loop in an $\mathfrak{su}(N)$ representation $R$
\begin{equation}
    W_R = \Tr_R \mathcal{P} \exp \left( i \oint A  \right)
\end{equation}
where the integral is taken over periodic Euclidean time. The thermal expectation value of this operator is an order parameter for confinement in pure gauge theories. In \cite{Witten:1998zw}, confinement in strong coupling limit of gauge theories was studied using dual gravity. Monte Carlo techniques have recently been used to confirm this M-theory picture \cite{Bergner:2021goh}. 