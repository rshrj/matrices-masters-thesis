In this thesis, we have presented original results on the classical real-time simulations of the bosonic sector of the BFSS matrix model. The results presented can be briefly summarized as follows
\begin{enumerate}
    \item[a.] The matrix model is highly chaotic. This is not a new result (see \cite{Berkowitz:2016znt,Gur-Ari:2015rcq}), but what we observe is something really interesting: the nature of sensitivity to initial conditions in this model is such that states very close to each other get farther and farther as time evolves until a certain time $t_S$ when the distance saturates and remains stable for all future times.
    \item[b.] Long-time statistics of many low-order observables such as $\Tr X^i X^i$ and the eigenvalues of the matrices are by-and-large independent of the initial condition one starts at, as long as we work in a constant energy surface. The dependence of various statistics on the energy of the state follows very simply from scaling arguments of \cref{eqn:scaling-similarity}.
    \item[c.] If one stares at these statistics for long enough, one finds an interesting connection to random matrices: the time distribution of the matrix variables look as if they were taken identically and independently from a traceless Gaussian Unitary Ensemble with a scale parameter $\alpha$ that depends only on the rank $N$ of the gauge group and the energy $E$ (in a simple manner due to scaling) of the state. At fixed energy, the scale parameter almost goes like $\alpha_N \sim \sqrt{N}$.
\end{enumerate}
Random matrix theory has been very successful at describing late-time horizon dynamics of certain black holes \cite{Cotler:2016fpe}. The fact that we find random matrix behavior in the matrix model and the typicality of states suggests that classical time-dependent coherent states in matrix theory somehow mimic individual microstates of the dual black hole.

\section{Future plans}

In the future, we plan to investigate if including the fermionic matrices $\theta$ changes the picture we see. The states we study break almost all of the supersymmetry the original BFSS model has (16 supercharges). Could time-dependent SUSY states offer new results not captured by the bosonic model we have been studying?

Secondly, we plan to include quantum corrections to the classical picture outlined in the thesis. To do this, we plan to use generalized coherent states highly localized on the classical solutions we have studied and look at how quantum mechanical corrections change the time evolution of such states and their spread in the space of solutions. This will allow us to make contact with energy regimes in which some semiclassical supergravity results might still be valid.

The results outlined in this dissertation are a small step towards a broader understanding of the emergence of space-time from quantum mechanical matrix models. Interpretations simplify at large $N$, but an interesting open question is what to make of the effective geometry at finite $N$. Is it a non-commutative dynamical space-time? Much remains to be explored.