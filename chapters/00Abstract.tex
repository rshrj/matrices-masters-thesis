Holographic duality has been very successful at studying strongly-coupled gauge theories by mapping the problem to one of classical gravity. It is interesting to ask whether (and what) one can learn about quantum gravity by studying gauge theories in certain approximations. 

In this dissertation, we present a summary of results in the study of the BFSS matrix model at high energies or, equivalently, weak coupling, where the dual gravity picture is highly stringy. The bosonic sector is analyzed through classical, real-time numerical simulations for a range of values of $N$ (the size of matrices). We find sensitivity to initial conditions of a particular nature that suggests the following picture: typical (random) classical states of matrix theory mimic microstates of the dual gravitational theory. 

Another feature reminiscent of black hole physics is random matrix behavior. The long-time distribution of matrices evolving through classical BFSS equations of motion is shown to resemble a traceless Gaussian Unitary Ensemble (GUE). We carefully match parameters on both sides (the matrix size $N$, the energy $E$ of the state considered, and a scale parameter $\alpha$ of GUE). Lastly, we present evidence suggesting higher-order correlations than those arising from simple random matrix ensembles. We discuss possible causes and fixes.